\documentclass[]{article}
\usepackage{lmodern}
\usepackage{amssymb,amsmath}
\usepackage{ifxetex,ifluatex}
\usepackage{fixltx2e} % provides \textsubscript
\ifnum 0\ifxetex 1\fi\ifluatex 1\fi=0 % if pdftex
  \usepackage[T1]{fontenc}
  \usepackage[utf8]{inputenc}
\else % if luatex or xelatex
  \ifxetex
    \usepackage{mathspec}
  \else
    \usepackage{fontspec}
  \fi
  \defaultfontfeatures{Ligatures=TeX,Scale=MatchLowercase}
\fi
% use upquote if available, for straight quotes in verbatim environments
\IfFileExists{upquote.sty}{\usepackage{upquote}}{}
% use microtype if available
\IfFileExists{microtype.sty}{%
\usepackage{microtype}
\UseMicrotypeSet[protrusion]{basicmath} % disable protrusion for tt fonts
}{}
\usepackage[margin=1in]{geometry}
\usepackage{hyperref}
\hypersetup{unicode=true,
            pdftitle={Recapture rates and habitat associations of White-faced Darter Leucorhinnia dubia on Fenn's and Whixall Moss, Shropshire, UK},
            pdfauthor={Rachel Davies, Achaz von Hardenberg \& Matthew Geary},
            pdfborder={0 0 0},
            breaklinks=true}
\urlstyle{same}  % don't use monospace font for urls
\usepackage{graphicx,grffile}
\makeatletter
\def\maxwidth{\ifdim\Gin@nat@width>\linewidth\linewidth\else\Gin@nat@width\fi}
\def\maxheight{\ifdim\Gin@nat@height>\textheight\textheight\else\Gin@nat@height\fi}
\makeatother
% Scale images if necessary, so that they will not overflow the page
% margins by default, and it is still possible to overwrite the defaults
% using explicit options in \includegraphics[width, height, ...]{}
\setkeys{Gin}{width=\maxwidth,height=\maxheight,keepaspectratio}
\IfFileExists{parskip.sty}{%
\usepackage{parskip}
}{% else
\setlength{\parindent}{0pt}
\setlength{\parskip}{6pt plus 2pt minus 1pt}
}
\setlength{\emergencystretch}{3em}  % prevent overfull lines
\providecommand{\tightlist}{%
  \setlength{\itemsep}{0pt}\setlength{\parskip}{0pt}}
\setcounter{secnumdepth}{0}
% Redefines (sub)paragraphs to behave more like sections
\ifx\paragraph\undefined\else
\let\oldparagraph\paragraph
\renewcommand{\paragraph}[1]{\oldparagraph{#1}\mbox{}}
\fi
\ifx\subparagraph\undefined\else
\let\oldsubparagraph\subparagraph
\renewcommand{\subparagraph}[1]{\oldsubparagraph{#1}\mbox{}}
\fi

%%% Use protect on footnotes to avoid problems with footnotes in titles
\let\rmarkdownfootnote\footnote%
\def\footnote{\protect\rmarkdownfootnote}

%%% Change title format to be more compact
\usepackage{titling}

% Create subtitle command for use in maketitle
\newcommand{\subtitle}[1]{
  \posttitle{
    \begin{center}\large#1\end{center}
    }
}

\setlength{\droptitle}{-2em}
  \title{Recapture rates and habitat associations of White-faced Darter
\emph{Leucorhinnia dubia} on Fenn's and Whixall Moss, Shropshire, UK}
  \pretitle{\vspace{\droptitle}\centering\huge}
  \posttitle{\par}
  \author{Rachel Davies, Achaz von Hardenberg \& Matthew Geary}
  \preauthor{\centering\large\emph}
  \postauthor{\par}
  \predate{\centering\large\emph}
  \postdate{\par}
  \date{24 May 2018}


\begin{document}
\maketitle

\subsection{Abstract}\label{abstract}

\subsection{Introduction}\label{introduction}

There has been a marked decline in global biodiversity in the last
several decades, a decline which is expected to continue, and this has
been largely attributed to changes in land-use activities
{[}@sala\_global\_2000{]}.Land-use activities include agriculture,
forestry, creation of urban areas, and use of natural resources
{[}@foley\_global\_2005{]}. These activities have a huge impact on
environmental characteristics and often cause habitat loss and
fragmentation, contributing largely to the decline in global species
diversity {[}@holloway\_conervation\_2003{]}. As such management and
protection of habitats and populations is vital at both a local and
global scale {[}@foley\_global\_2005; @holloway\_conervation\_2003{]}.

A bias exists in conservation research towards charismatic vertebrates
{[}@di\_marco\_changing\_2017{]}. Although Odonata are charismatic
invertebrates they are not immune to this bias
{[}@clausnitzer\_odonata\_2009{]}. In addition much research into
Odonata focuses on physiology, evolution and behaviour
{[}@cordoba-aguilar\_dragonflies\_2008{]} and they have rarely been the
focus of conservation research {[}@clausnitzer\_odonata\_2009{]}. Basic
ecological research into demography, survival and habitat use is
essential for effective protection of species and habitats. For any taxa
this requires detailed ecological and life history data collected in the
field. These are often difficult to obtain, particularly on large
scales. Integrating large scale data such as presence-only distribution
datasets with more detailed local information is a current challenge in
conservation ecology {[}@powney\_beyond\_2015{]}.

Methods to analyse habitat preferences are varied depending on the data
available. The current `gold standard' is the use of site occupancy
models which take into account detectability (i.e.~the probability that
a species is detected in a site if present) when estimating occupancy:
the probability that a species is present in a site
{[}@mackenzie\_estimating\_2003{]}. Models using this framework help us
to avoid the age-old problem of ``imperfect detection'', i.e.~failing to
spot a species during a survey on a site where it is actually present
{[}@mackenzie\_estimating\_2003{]}. However, these models require
repeated surveys where both detections and non-detections are recorded;
these data are not always available, especially in the case of historic
data collected by volunteers. On larger scales a number of methods exist
which can use only presence records along with environmental covariates
{[}@elith\_species\_2009{]}. These can tell us about habitat use but are
constrained to estimate a measure of the relative importance of habitats
rather than the true probability of presence
{[}@elith\_statistical\_2011{]} and are limited by the environmental
data available. At very small scales, such as individual protected
areas, detailed data on habitats and landcover can be difficult to
obtain. Datasets such as the UK lancover map (LCM2015), although the
resolution is 15m, are too crude for local studies in some areas.
Simpler methods which indicate preferred habitat, such as selection
indices {[}@manly\_resource\_2007{]}, have fewer assumptions and can be
revealing even at small scales {[}@neu\_technique\_1974{]}.

Investigating survival and movement requires individual recognition and
methods using a capture mark-recapture approach are well established
{[}@mccrea\_analysis\_2014{]}. Such analyses can tell us about the
age-sex specific survival probabilites of individuals, the use of
different sites or habitats and how these changes over time and the
likelihood of encountering individuals again in the future. High quality
data of this type can provide accurate estimates of population size.
Mark-recapture methods have been used on Odonata populations in the past
to monitor rare species as well as to study the life history of more
abundant species {[}@oster\_evaluating\_2004;
@cordero-rivera\_mark-recapture\_2008{]} to monitor rare species as well
as to study the life history of more abundant species
{[}@anholt\_mark-recapture\_2001{]}. Odonata, because of the ease by
which they can usually be individually marked, have also been used as
model species for methodological research on the development of
mark-recapture techniques {[}@manly1968{]}.

The White-faced Darter (\emph{Leucorrhinia dubia}), is a specialist of
lowland peatbogs where it breeds in bog pools containing sphagnum mosses
{[}@smallshire\_britains\_2004{]}. It has a life cycle that includes a
1-3-year larval period, followed by an adult flight period
{[}@smallshire\_britains\_2004{]}. Emergence is weather dependent and
will typically start in either May or June each year. Tenerals are
thought to disperse to low scrub following emergence, staying there
whilst they mature. Following this, the adults return to breeding pools,
with males returning sooner than females so they can hold breeding
territories {[}@smallshire\_britains\_2004{]}. The adult flight period
typically ends in either late July or August. The White-faced Darter has
a scattered distribution and its populations have been declining in
Britain over the past several decades. Despite being classified as a
species of least concern on the IUCN Red Data List
{[}@clausnitzer\_odonata\_2009{]}, this decline in Britain has resulted
in a classification of Endangered on the Odonata Red Data List for
Britain {[}@beynon\_odonata\_2008{]}. This decline is largely attributed
to habitat loss and the resulting habitat fragmentation
{[}@beynon\_odonata\_2008{]}; as over 90\% of England's peatbogs have
been lost or substantially damaged to date {[}@nature\_peat\_2002{]}.
There are currently only three stable historical populations of
White-faced Darter in England, along with two recently reintroduced
populations, one in Cumbria and one in Cheshire
{[}@clarke\_white-faced\_2014; @meredith\_reintroduction\_2017{]}.

Here we use two methods to investigate important ecological
characteristics of White-faced Darter on Fenn's and Whixall moss in
Shropshire, UK. We use mark-recapture methods to investigate survival
and movements of adults during the flight period and a selection index
method to investigate habitat use. These methods can both contribute to
our understanding of the spatial use of habitat by White-faced Darter
and can help us to prioritise future research for this species.

\subsection{Methods}\label{methods}

\subsubsection{Study area}\label{study-area}

Fenn's, Whixall and Bettisfield Mosses (FWB Mosses) are located within
Shropshire (52°55′N 2°46′W) and they support a large, long-established
population of White-faced Darter. FWB Mosses are a lowland raised bog
complex, stretching nearly 1000 hectares
{[}@meredith\_reintroduction\_2017{]}. Historically, the mosses were
used for peat cutting and in the 19 th century they were drained to
allow larger-scale operations to take place
{[}@meredith\_reintroduction\_2017{]}. Eventually, in 1990, the mosses
were taken over by English Nature (now Natural England) and long-term
restoration began, benefitting a whole host of mossland species,
including the White-faced Darter. {[}@meredith\_reintroduction\_2017{]}.

\subsubsection{Field methods}\label{field-methods}

The site was surveyed twice per week between the 22nd of May and 6th of
July 2017. This encompassed the peak flight period of White-faced Darter
{[}@smallshire\_britains\_2004{]}. Two separate breeding pools within
FWB Mosses were sampled simultaneously, along with a variety of scrub
and other potentially suitable habitat. On each sampling occasion, the
full sampling area was searched for any White-faced Darter individuals.
Different routes were walked on each occasion to allow different areas
within the sampling area to be searched at different times of the day.
Sampling sessions lasting between 5-10 hours, being carried out between
10am and 4pm, as this is the favoured flight period for adult
dragonflies {[}@smallshire\_dragonfly\_2010{]}. Sampling days were
weather dependent {[}@chin\_interactive\_2009{]} and weather conditions
were recorded on all sampling days.

\emph{Capture-Mark-Recapture}\\
Where possible, mature adults were caught using an invertebrate net and
marked with a unique number on their wing
{[}@chin\_interactive\_2009{]}, using an Edding 404 permanent marker pen
(Plate.1 - NEED TO ADD IN PICTURE - IS THAT OK RACHEL?). The insects
were then released at point of capture and any behavioural observations
recorded. Not all observed individuals were captured and tenerals were
excluded from the capture-mark-recapture survey as during this life
stage they are fragile and handling may cause wing damage
{[}@allen\_population\_2014{]}. Tenerals are easily identified by their
pale green colouration, a lack of their full adult colouration and by
their shiny wings {[}@smallshire\_britains\_2004{]}. Insects recaptured
on day of marking were not re-counted {[}@foster\_evaluating\_2004{]}.
Following an initial marking, recapture on successive days was only
necessary when relevant information could not be collected from
re-sighting individuals {[}@lettink\_introduction\_2003{]}.

\emph{Selection index} White-faced Darter presence was recorded while
searching the site during the capture-mark-recapture study. This
included captured individuals as well as those seen on survey routes but
not captured. On each occasion the location of the individual was
recorded with a hand-held GPS unit (Garmin GPSMAP 64). Additionally, a
phase 1 habitat survey {[}@committee\_handbook\_2010{]} was conducted
across the study site to produce a habitat map using 100 x 100 m grid
cells. The proportion of five habitat types were recorded in each
square: moss (peat moss, rushes and sedges), scrub (low woody
vegetation), scrub-moss (peat moss with low woody vegetation), water
(open pools) and woodland (mature trees). From this the dominat habitat
in each square was calculated. Of these, only water was not used in
analyses as adult individuals tended to be sighted over terrestrial
habitat.

\subsubsection{Data analysis}\label{data-analysis}

\emph{Capture-mark-recapture}\\
A single-season Cormack-Jolly-Seber model
{[}@pradel\_utilization\_1996{]} was used to determine to probability of
survival between sampling days and the probability of recapture on
sampling days for the capture-mark-recapture data.
Capture-mark-recapture analysis was carried out using the `RMark'
package version 2.2.2 {[}@fouchet\_r\_2015{]}, in R version 3.5.0
{[}@Manual\_R{]}.

\emph{Selection index}\\
Selection indices calculate habitat use as a ratio between habitat where
a species is recorded compared to the proportion of each habitat within
the study area {[}@manly\_resource\_2007{]}. Although relatively simple
they can be effective in indicating habitat use
{[}@manly\_resource\_2007{]}. Selection indices can be sensitive to the
scale used in calculating habitat use however Neu's index is relatively
robust to changes in scale {[}@neu\_technique\_1974{]}. For this reason
we used Neu's index which calculates \(w_{i} = \frac{u_{i}}{\pi_{i}}\)
where \(w_{i}\) is the proportion of squares of each dominant habitat
type among all of the squares wiht White-faced Darter records and
\(\pi_{i}\) is the proportion of each dominat habitat type among all of
the squares in the study area. Values of thsi index \(> 1\) indicate use
of a habitat type in greater proportions than it is generally available
in the study area. Selection index analysis was performed in R version
3.5.0 {[}@manual\_R{]}.

\subsection{Results}\label{results}

\emph{Capture-Mark-Recapture Model}

\begin{verbatim}
## Building constraints matrices 
## Starting optimization
\end{verbatim}

A total of 13 sampling days were carried out at FWB Mosses from the 22nd
May 2017 until the 7th July 2017. During these sampling days, a total of
50 adult White-faced Darter were marked (41 males, 9 females), and a
total of 6 recaptures were made (fig!!). Probability of survival between
sampling days was estimated at \texttt{s\_back} (95\% confidence
intervals: \texttt{s\_CI95L}-\texttt{s\_CI95U}). Probability of capture
on each sampling day was estimated at 0.39 (SE = 0.76, 95\% confidence
intervals: 0.06, 0.86). Further models using a range of co-variates were
unsuitable as the models became over parameterised due to the lack of
recaptures.

\includegraphics{WFD_FW_Moss_files/figure-latex/capture_mark_recapture_figure_1-1.pdf}
\includegraphics{WFD_FW_Moss_files/figure-latex/capture_mark_recapture_figure_1-2.pdf}

\includegraphics{WFD_FW_Moss_files/figure-latex/capture_mark_recapture_figure_2-1.pdf}

\includegraphics{WFD_FW_Moss_files/figure-latex/capture_mark_recapture_figure_3-1.pdf}

\includegraphics{WFD_FW_Moss_files/figure-latex/capture_mark_recapture_figure_4-1.pdf}

\emph{Selection Index} A further 304 individual White-faced Darter were
observed during the fieldwork, 234 of which were not captured but only
observed from distance (Figure!!!). White-faced Darter show a clear
preference (SI \textgreater{} 1) for `moss' habitats while scrub and
woodland (smallest SI) appear to be avoided (Figure!!!).

\includegraphics{WFD_FW_Moss_files/figure-latex/selection_index_figure-1.pdf}

\subsubsection{Discussion}\label{discussion}

The capture-mark-recapture model suggested that the survival rate of
adult White-faced Darter from one survey session to the next was 25\%,
however, the confidence intervals around this estimate were very wide.
Similarly, we estimated a 40\% chance of each individual being
recaptured by our methods but with very wide confidence intervals. The
reason for the wide confidence intervals around these estimates was the
very low recapture rate (only 6 recaptures in 13 survey days). Although
low capture rates might be expected in a large invertebrate population
and have been noted before in Odonata
{[}@cordero-rivera\_mark-recapture\_2008{]}, this was lower than
expected. Although male White-faced Darter hold territories they are
less tied to these sites than species such as Four-spotted chaser
(\emph{Libellula quadrimaculata}) and so are less predictable in their
movements {[}@merritt1996atlas{]}. We suggest that in future
capture-mark-recapture approaches for this species, and other similarly
cryptic species, need a greater number of capture days and more
researchers in the field making captures. This increase in effort is
likely to increase the capture rate and increase the accuracy of
estimates.

Many more White-faced Darter were seen than were captured and the
resource selection index calculated using these data suggest that they
prefer the `moss' habitat among those available. Although this habitat
is the most common habitat in the study area, the selection index
suggests that they use this habitat in proportions greater than those
available across the site. The `moss' habitat consists of peat with low
heather vegetation and wet flushes and is the habitat most commonly
found at pool edges. This is the habitat described in previous research
on White-faced Darter {[}@dolny\_how\_2018{]} and described in Boudot \&
Kalman {[}@boudot\_atlas\_2015{]} including ``peat moss, rushes and
sedges''. Locally on this site, White-faced Darter appear to avoid
complex vegetation, including scrub and woodland. However, White-faced
Darter sites, especially those in Scotland which represents the
stronghold for this species in Britain, are often forested
{[}@cham2014atlas{]}. Breeding pools within these sites are likely to be
in open areas but the association with woodland, particularly ancient
woodland {[}@cham2014atlas{]}, is suggestive of some associations
between White-faced Darter and these habitats at larger scales.

Moss habitats are certainly suitable for White-faced Darter, however,
the low capture and recapture rate we found in this study may explain
why open `moss' habitats appeared to be preferred for this species.
White-faced Darter are well camouflaged within their habitats and, as
such, there is a good chance of missing individuals because of habitat
complexity {[}i.e. low detectability, @mazerolle\_making\_2007{]}.
Unfortunately, our field methods, did not allow us to estimate
detectability in terms of sightings on this occasion but the low capture
probability found in our capture-mark-recapture study suggests it is
very low. In future we suggest that survey methods are designed so that
detectability can be estimated explicitly, in order to get more accurate
estimates of occupancy and thus of resource selection. In this case we
are left unable to confidently suggest whether White-faced Darter are
avoiding more complex vegetation or whether they are harder to see and
therefore record in these habitats.

Data which allows the estimation of detectability can easily be
collected with just a few minor changes to currently common survey
methods. In fact, the majority of these suggestions are already being
requested by the BDS to provide data for the upcoming State of the
Nation's Dragonflies in 2020. We would like to add our voice to these
calls to record complete lists and to repeat site visits. Complete lists
are records of all the Odonata species detected on a single visit and
allow non-detection to be inferred where species are not recorded
{[}@isaac\_bias\_2015{]}. This requires recorders to note very common
species as well as rarities. Unfortunatley, there is a tendency in
biological recordings to note only the rare or exciting species
(e.g.~first record of the year) and this can bias our inferences about
population change amongst more common species
{[}@isaac\_statistics\_2014{]}. Repeated site visits allow us to
estimate the detectability of a species
{[}@mackenzie\_estimating\_2003{]} and consequently get unbiased
estimates of occupancy, not affected by imperfect detection. We would
also like to suggest that where possible recorders include some measure
of effort in their surveys (e.g.~time spent surveying or distance
walked). Ideally this would be standardized and included in official
protocols such as those already commonly in use for bird surveys.
However, this suggestion is less vital because effort could potentially
be crudely estimated through list-length analysis
{[}@szabo\_regional\_2010{]} provided that multiple visits are made to
the same sites .

We present the results in this paper as an indication of what can be
done in terms of conservation research in Odonata. Although we have been
unable to make firm inferences regarding White-faced Darter survival and
habitat preference at this stage this study can provide valuable
information which can contribute to the design of future studies. We
suggest that research into the conservation ecology of White-faced
Darter along with other Odonata species threatened with declining
ranges, declining populations or habitat loss is essential to the
long-term conservation of these species. Methods for such studies can be
well informed by current practices used with other taxa. In particular,
the analytical advances made in ornithology, research on Lepidoptera and
work related to the use of data collected through citizen science
provide a fantastic opportunity to advance our knowledge on the
conservation ecology of Odonata.

\subsubsection{Acknowledgements}\label{acknowledgements}

We would like to express our thanks to the British Dragonfly Society who
supported this work financially, Natural England who provided permission
to access the field site and to Chris Meredith who provided invaluable
help in the field and with advice on White-faced Darter ecology and
behaviour. We would also like to thank all of the volunteers who helped
with fieldwork in the summer of 2017.

\subsubsection{References}\label{references}


\end{document}
